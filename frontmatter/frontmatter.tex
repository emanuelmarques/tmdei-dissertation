
% we include the glossary here (frontmatter is included with \input, so this command is as if it was in main.tex)
%All acronyms must be written in this file.
\newacronym{SP}{SP}{Service Provider}
\newacronym{PoC}{PoC}{Proof of Concept}
\newglossaryentry{API}
{
  name={API},
  description={Application Programming Interface},
  first={\glsentrydesc{API} (\glsentrytext{API})},
  plural={APIs},
  descriptionplural={Application Programming Interfaces},
  firstplural={\glsentrydescplural{APIs} (\glsentryplural{APIs})}
} 
\newacronym{AoA}{AoA}{Area of Actuation}
\newacronym{SDP}{SDP}{Service Display Page}
\newacronym{SLP}{SLP}{Service Listing Page}
\newacronym{SPLP}{SPLP}{Service Provider Listing Page}
\newacronym{MVP}{MVP}{Minimum Viable Product}
\newacronym{SOA}{SOA}{Service Oriented Architecture}
\newacronym{SQL}{SQL}{Structured Query Language}
\newacronym{CQRS}{CQRS}{Command Query Responsibility Segregation}
\newacronym{OS}{OS}{Operating System}
\newacronym{CRUD}{CRUD}{Create, Read, Update and Delete}
\newacronym{DDD}{DDD}{Domain Driven Design}
\newacronym{ESB}{ESB}{Enterprise Service Bus}
\newacronym{FCL}{FCL}{Framework Class Library}
\newacronym{CLR}{CLR}{Common Language Runtime}
\newacronym{IDE}{IDE}{Integrated Development Environment}
\newacronym{ASP}{ASP}{Active Server Pages}
\newacronym{AOP}{AOP}{Aspect Oriented Programming}
\newacronym{JPA}{JPA}{Java Persistence API}
\newacronym{JVM}{JVM}{Java Virtual Machine}
\newacronym{SRP}{SRP}{Single Responsibility Principle}
\newacronym{FFE}{FFE}{Fuzzy Front End}
\newacronym{NCD}{NCD}{New Concept Development}
\newacronym{NPD}{NPD}{New Product Development}
\newacronym{FEI}{FEI}{Front End of Innovation}
\newacronym{NPPD}{NPPD}{New Product and Process Development}
\newacronym{ROI}{ROI}{Return on Investment}
\newacronym{SRS}{SRS}{Software Requirements Specification}
\newacronym{VC}{VC}{Value for the Customer}
\newacronym{CVP}{CVP}{Customer Value Proposition}
\newacronym{PoP}{PoP}{Point of Parity}
\newacronym{QEF}{QEF}{Quality Evaluation Framework}
\newacronym{SAAM}{SAAM}{Software Architecture Analysis Method}
\newacronym{SEI}{SEI}{Software Engineering Institute}
\newacronym{CBAM}{CBAM}{Cost-Benefit Analysis Method}
\newacronym{HTTP}{HTTP}{Hypertext Transfer Protocol}
\newacronym{UWP}{UWP}{Universal Windows Platform}
\newacronym{ORM}{ORM}{Object-Relational Mapper}
\newacronym{XML}{XML}{Extinsible Markup Language}
\newacronym{IoC}{IoC}{Inversion of Control}
\newacronym{JDK}{JDK}{Java Development Kit}
\newacronym{JS}{JS}{JavaScript}
\newacronym{I/O}{I/O}{Input/Output}
\newacronym{BO}{BO}{Back-Office}
\newacronym{FO}{FO}{Front-Office}
\newacronym{GPS}{GPS}{Global Positioning System}
\newacronym{CPU}{CPU}{Central Processing Unit}
\newacronym{RAM}{RAM}{Random Access Memory}
\newacronym{AHP}{AHP}{Analytic Hierarchy Process}
\newacronym{CI}{CI}{Continuous Integration}
\newacronym{CD}{CD}{Continuous Delivery}
\newacronym{KLOC}{KLOC}{Thousand Lines Of Code}
\newacronym{DTO}{DTO}{Data Transfer Object}
\newacronym{DBO}{DBO}{Database Object}
\newacronym{UI}{UI}{User Interface}
\newacronym{GUI}{GUI}{Graphic User Interface}
\newacronym{DLL}{DLL}{Dynamic Link Library}
\newacronym{VIP}{VIP}{Very Important Person}
\newacronym{VAT}{VAT}{Value Added Tax}
\newacronym{UX}{UX}{User Experience}
\newacronym{VCS}{VCS}{Version Control System}


\frontmatter % Use roman page numbering style (i, ii, iii, iv...) for the pre-content pages

\pagestyle{plain} % Default to the plain heading style until the thesis style is called for the body content

%----------------------------------------------------------------------------------------
%	TITLE PAGE
%----------------------------------------------------------------------------------------

\maketitlepage

%----------------------------------------------------------------------------------------
%	DEDICATION  (optional)
%----------------------------------------------------------------------------------------
%
%\dedicatory{For/Dedicated to/To my\ldots}
\begin{dedicatory}
This work is dedicated to my parents, for all the work they had with me and everything they taught me. They made me who I am today and I will be forever grateful for the excellent education they gave me.

My greatest sincere thank you!

\end{dedicatory}

\begin{dedicatoryotherlanguage}
Este trabalho é dedicado aos meus pais, por todo o trabalho e dedicação que tiveram comigo e tudo o que me ensinaram. É graças a eles que hoje sou quem sou, e estarei para sempre grato pela excelente educação que eles me deram.

O meu mais sincero obrigado!
\end{dedicatoryotherlanguage}



%----------------------------------------------------------------------------------------
%	ABSTRACT PAGE
%----------------------------------------------------------------------------------------

\begin{abstract}
Society is increasingly tied to technology. If you look around you, you see countless pieces of engineering to which people are becoming more and more accustomed and outstanding. This is due to the efficiency with which technological solutions respond to the needs of their users. Nowadays it is possible to do almost everything online: buying clothes, traveling, ordering food and being delivered at home, are examples of things that were unthinkable 30 years ago. 
\par
The e-commerce brought a revolution that has shaken many industries. From music to retail, through to services, there was none to which the appearance of this new kind of business was indifferent. The possibility of selling online to any part of the world made it possible for even small businesses to export and compete with large companies. 
\par
However, the application of this type of trade to services is not as strong as when compared to the sale of products. There are many more platforms for selling products via the Internet than platforms for providing services. 
\par
With this, some companies started to provide this service, but without using technology. All the management of orders and receipts is done by telephone, and the records made in paper or on an Excel sheet, which causes employees to spend time on tasks that are not their area of expertise. 
\par
This thesis aims at creating a platform to meet these needs by providing a range of tools to facilitate the work of service providers. To achieve this, a platform capable of supporting multiple service types and multiple marketplace service providers will be designed and developed.

\par
This platform will be composed of several applications, from the platform's main site and back-office application, to mobile applications for couriers. These are aimed at a segmented response to the needs of the three existing types of customer: service providers, remittance agents and, of course, the final customer. 
\par
In this context, a laundry will be used as the pilot, which will lead to the evaluation and testing phase.

\end{abstract}

\begin{abstractotherlanguage}
A sociedade está cada vez mais ligada à tecnologia. Se olharmos à nossa volta, vemos inúmeras peças de engenharia, às quais as pessoas estão cada vez mais habituadas e dependentes. Tal acontece devido à eficiência com que as soluções tecnológicas respondem às necessidades dos seus utilizadores. Nos dias de hoje é possivel fazer praticamente tudo via internet: comprar roupa, viagens, encomendar comida e esta ser entregue em casa, são exemplos de coisas que há 30 anos era impensável que se conseguisse.
\par
O e-commerce trouxe uma revolução que abanou muitas indústrias. Da música ao retalho, passando pelos serviços, não houve nenhuma ao qual o aparecimento deste novo tipo de comércio fosse indiferente. A possibilidade de vender online para qualquer parte do mundo possibilitou que mesmo pequenos negócios pudessem exportar e competir com as grandes empresas. 
\par
No entanto, a aplicação deste tipo de comércio a serviços, ainda não é tão forte como quando comparado à venda de produtos. Existem muito mais plataformas de venda de produtos via internet, do que plataformas de prestação de serviços.
\par
Com isto, algumas empresas começaram a prestar esse serviço, mas sem recurso à tecnologia. Toda a gestão de pedidos e estafetas é feita via telefone, e os registos feitos em papel ou numa folha de Excel, o que faz com que os funcionários gastem tempo em tarefas que não são a sua área de especialização.
\par
Esta tese visa a criação de uma plataforma que dê resposta a estas necessidades, providenciando uma série de ferramentas que facilite o trabalho dos prestadores de serviços. Para tal, será idealizada e desenvolvida uma plataforma capaz de suportar múltiplos tipos de serviços e múltiplos prestadores de serviços, ao estilo marketplace. 
\par
Esta plataforma será composta por várias aplicações, desde o site principal da plataforma e aplicação back-office, até aplicações móveis para os estafetas. Estas têm como objetivo dar uma resposta segmentada às necessidades dos três tipos de cliente existentes: os prestadores de serviços, os estafetas e, obviamente, o cliente final.

\par
Neste contexto, será tido como cliente piloto uma lavandaria, que conduzirá a fase de testes e avaliação.


\end{abstractotherlanguage}

%----------------------------------------------------------------------------------------
%	ACKNOWLEDGEMENTS (optional)
%----------------------------------------------------------------------------------------

\begin{acknowledgements}
Since this is very likely to be the end of my academic education, I feel that I should take this opportunity to thank all the people who have come across me over the last eighteen years of study and have, somehow, contributed to my academic, professional and personal education.
\par
I begin by thanking all the elementary and secondary school teachers who taught me the various disciplines of knowledge. They showed me how the world has come to this day and taught me how to speak a language without which I couldn't communicate with most people on this planet. To do head counts in order to gain logical reasoning and proved me that every action has a reaction. Essentially they taught me to think.

\par

The various professors I came across during this journey at Instituto Superior de Engenharia do Porto, who not only taught me how to be a software engineer, taught me how to be a good engineer, showing me the best practices of the industry and taught me to have a critical attitude when faced with new problems.

\par

To all my colleagues with whom I took this walk side by side, especially the colleagues I came across in designing and developing the Núcleo de Estudantes de Informática do ISEP. They allowed the development of a range of skills that are not possible to be learned in the classroom.

\par

To all my family, who has always been by my side, to celebrate the good times and to hug me in the least good ones. The fact is, I'm lucky to have such people around me.

\par
To my thesis advisor, Prof. António Rocha for, besides having taught me a lot in the classes in which he was my teacher, he accepted to guide me through this challenge.

\par

To the fantastic people I came across at Farfetch for helping me grow so much as a professional over so little time. A very special thank you to Fernando Costa for being my adopted mentor and for continuing to be a great colleague and friend who readily offered to support me in this project.

\par

To everyone who somehow contributed to my professional success in one way or another without you was not the person I am today.


\par
Thank you very much.

\par

\begin{flushright}
   \underline{\textit{Emanuel Marques}}
\end{flushright}


\end{acknowledgements}

\begin{acknowledgementsotherlanguage}
Sendo este o muito provável terminar da minha formação académica, sinto que devo aproveitar esta oportunidade para agradecer a todas as pessoas que se foram cruzando comigo ao longo desta caminhada de mais de dezoito anos de estudo e de alguma forma contribuíram para a minha formação académica, profissional e pessoal.
\par
Começo por agradecer a todos os professores do ensino básico e secundário, que me ensinaram as várias disciplinas do conhecimento. Que me mostraram como o mundo chegou aos dias de hoje e me ensinaram a falar uma língua sem a qual não conseguia comunicar com a maioria das pessoas deste planeta. Que me ensinaram a fazer contas de cabeça de forma a conseguir obter um raciocínio lógico e me provaram que cada ação tem uma reação. Em suma, ensinaram-me a pensar.

\par

Aos vários professores com os quais me fui cruzando ao longo desta jornada pelo Instituto Superior de Engenharia do Porto, que não só me ensinaram a ser engenheiro de software, ensinaram-me a ser um bom engenheiro, mostrando-me as melhores práticas da indústria e a ter uma atitude crítica quando confrontado com novos problemas.

\par

A todos os meus colegas com os quais dei esta caminhada lado a lado, em especial aos colegas com que me cruzei na conceção e crescimento do Núcleo de Estudantes de Informática do ISEP. Estes permitiram o desenvolvimento de uma série de competências que não se aprende em sala de aula.

\par

A toda a minha família, que sempre esteve ao meu lado, para celebrar os bons momentos e para me abraçar nos menos bons. Facto é que sou um sortudo por ter tais pessoas perto de mim.

\par
Ao meu orientador de tese, Prof. António Rocha por, para além de muito me ter ensinado nas aulas que foi meu professor, ter aceite orientar este desafio.

\par

Às fantásticas pessoas com que me cruzei na Farfetch por terem-me ajudado a crescer tanto enquanto profissional num espaço de tempo. Um obrigado muito especial ao Fernando Costa, por ter sido o meu mentor adotado e por continuar a ser um grande colega e amigo que prontamente se disponibilizou para me apoiar neste projeto.

\par 

A todos os que de alguma forma contribuíram para o meu sucesso profissional, de uma forma ou de outra, sem vocês não era a pessoa que sou hoje. 


\par
Muito obrigado.

\par

\begin{flushright}
   \underline{\textit{Emanuel Marques}}
\end{flushright}

\end{acknowledgementsotherlanguage}

%----------------------------------------------------------------------------------------
%	LIST OF CONTENTS/FIGURES/TABLES PAGES
%----------------------------------------------------------------------------------------

\tableofcontents % Prints the main table of contents

\listoffigures % Prints the list of figures

\listoftables % Prints the list of tables

%\iflanguage{portuguese}{
%\renewcommand{\listalgorithmname}{Lista de Algor\'itmos}
%}
%\listofalgorithms % Prints the list of algorithms
%\addchaptertocentry{\listalgorithmname}


\renewcommand{\lstlistlistingname}{List of Source Code}
\iflanguage{portuguese}{
\renewcommand{\lstlistlistingname}{Lista de C\'odigo}
}
\lstlistoflistings % Prints the list of listings (programming language source code)
\addchaptertocentry{\lstlistlistingname}


%----------------------------------------------------------------------------------------
%	ACRONYMS
%----------------------------------------------------------------------------------------

\newcommand{\listacronymname}{List of Acronyms}
\iflanguage{portuguese}{
\renewcommand{\listacronymname}{Lista de Acr\'onimos}
}

%Use GLS
\glsresetall
\printglossary[title=\listacronymname,type=\acronymtype,style=long]

%----------------------------------------------------------------------------------------
%	DONE
%----------------------------------------------------------------------------------------

\mainmatter % Begin numeric (1,2,3...) page numbering
\pagestyle{thesis} % Return the page headers back to the "thesis" style
